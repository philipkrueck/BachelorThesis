
\chapter{Results} \label{chapter::results}
This chapter focuses on the evaluation of the chosen empirical methods for evaluating the research question (Section \ref{section::thesis_objective}).
Specifically, the executed performance profiling results for both apps
along the chosen metrics (see Subsection \ref{subsection::selected_measurement_variables}) are comparatively evaluated in order to test the validity of the performance hypothesis $H_P$ (Section \ref{section::methods}).
Furthermore, the conducted expert interviews are analysed using the process of interview coding (see Subsection \ref{section::interview_evaluation}) to corroborate the usability hypothesis $H_U$ (Section \ref{section::methods}).

\section{Performance Comparison} \label{section::performance_comparison}






\section{Usability Comparison} \label{section::usability_comparison}
This section presents and contextualizes the results from the expert interview coding process (see Section \ref{section::interview_evaluation}).
In most cases the expert interview participants did not prefer either of the two user experiences as can be seen in Table [Interview Qualitative Analysis].
Particularly, cases where no differences could be perceived include:
\begin{itemize}
    \item opening application
    \item vertical scrolling on overview page
    \item horizontal scrolling in image gallery
    \item switch control interaction
\end{itemize}
It is noteworthy to point out that scrolling through a list of UI items derived from a network based data source 
- noteworthy to point out that scrolling through a list of items derived from a network based data source may have been the expected bottleneck in terms of usability perception


- Ziel des Interview Coding -> differences herauskristallisieren
- Things where participants could not tell the differences
    - App Start
    - Vertical Scrolling on Overview page
    - Horizontal Scrolling in Image gallery in terms of smoothness
    - Switch Control interaction
- In many cases participants had to go through use cases multiple times in order to find differences between the two applications





- Kategorien wurden dementsprechend entwickelt
    - General differences
    - Detail Transition differences
    - Modal Transition differences
    - Textfield Interaction differences
    - Horizontal Scrolling differences
    - Webview Component Interaction differences
    - Other Specific Implementation differences