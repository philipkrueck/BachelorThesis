\chapter{Introduction}
\label{section:introduction}
Mobile platforms are dominated by two players - Apple and Google with their respective operating systems iOS and Android. 
Cumulatively, they form a duopoly in the smartphone operating systems market with a combined usage share of 
15.2\% for iOS and 84.8\% for Android in 2020 according to \textcite{IDC2021}.
\\To develop a mobile application for both target platforms, the corresponding development environments and technologies 
are utilized for each platform. This leads to a doubling of cost, development time, and 
the need for knowledge of two different application development paradigms. 
This has resulted in the creation of cross-platform frameworks such as Xamarin (\cite{Xamarin2021}), React Native (\cite{Facebook2021}), and Ionic (\cite{Ionic2021}). 

The fundamental principle behind these frameworks is a specific tech stack operating on a single code base leading to increased development speed
while also having the ability to deploy for both operating systems.

Generally, cross-platform frameworks utilize a software bridge to communicate with the underlying host platform plugging into 
native interfaces. During app runtime, transmission delays may occur leading to reduced execution speed. In addition, 
cross-platform frameworks provide abstract interfaces over multiple native interfaces leading to a decreased subset of functionality
and especially impaired UI customizability. These inherent architecture attributes (further see Section \ref{section::other_architectures}) explain
both impeded performance (\cite{Ebone2018}) and user experience (\cite{Mercado2016}) compared to native technologies. 

Flutter (\cite{FlutterDev20}) is a newly developed open-source UI toolkit developed by Google. 
It has a nonconventional approach to cross-platform development in that every app ships with the Flutter rendering engine. 
Thereby, Flutter bypasses host platform communication in terms of UI rendering and delivers a natively compiled binary 
which can be executed by the underlying device (see Section \ref{section::flutter_architecture}).
Furthermore, the framework provides a \textit{"[...] collection of visual, structural, platform, and interactive widgets"} (\textcite{GoogleWidgets2021}) for UI customization.
It seems that Flutter addresses the very issues that are generally criticized about cross-platform solutions.

Unfortunately, since Flutter was first released in March 2018 (\cite{FlutterReleases2020}), 
there are no peer reviewed articles comparing the performance or usability to native apps\footnote{A search for relevant articles has been conducted using Google Scholar, Sci-hub and IEEE Xplore.}. 

\section{Motivation}
As a digital agency specialized on native iOS and Android development, apploft GmbH 
(the partnering company of this thesis) is highly interested in Flutter. 
The implications of using this framework could be wide ranging including the extension of the services portfolio
to clients with lower budgets.
For example, startups which are unsure about product market-fit and held by tight budget constraints are especially keen on reaching 
the maximum number of potential customers with their app. Flutter could be utilized for a fast iteration of a product deployable to 
both mobile platforms. 
Worth noting are the benefits of serving smaller customers like startups which include higher growth potential for long-term cooperation
and risk diversification in apploft's client portfolio.\\
The potential of Flutter goes beyond acquiring small clients. Instead of focusing on platform customization with native tooling, more effort could 
be directed into developing unique custom features.
Furthermore, infrastructure setup, package development and app updates would only be necessary for one codebase.\\
The above stated possible implications are not only relevant for apploft, but mobile application developers at large.

\section{Thesis Goal}
The aim of this thesis is derived based on the above stated problems with cross-platform frameworks (section ..) and the potential business opportunities mentioned in section.
Specifically, Google's claim that Flutter can match "native performance" and the framework can be utilized for building "expressive and flexible UI" (cite flutter.io)
will both serve as inductively derived hypotheses that shall be empirically verified or falsified individually by this thesis.


%% METHODS
The methodology to evaluate the stated hypotheses will be to compare common use cases between a Flutter and native mobile application.
Typical mobile application features include the visual and interactive representation of information, horizontal and vertical scrolling, 
communication with a remote API, different means of navigation between screens 
as well as animations and transitions.
The feature selection process is further explained in Section \ref{section::feature_selection}.

Based on these characteristics, \textit{Kickdown} (\cite{Kickdown2021}) - an online car auction app - was chosen. 
The app has already been developed for iOS by apploft (link to appstore).
To verify laid out claims of the Flutter framework, an app with a relevant specific subset of matching functionality
and UX design is reproduced.

\subsection{Performance comparison}
To evaluate performance, the typical measures of CPU, GPU and memory usage are chosen in this thesis. 
On the one hand, these metrics are the underlying causes of more ephemeral metrics such as page load speed. 
On the other hand, they can be easily measured using software tools as explained in Section \ref{section::performance_comparison_design}.

\subsection{Usability comparison}
Expert interviews with employees of apploft are conducted to compare the user experience of the developed Flutter app with the iOS application.
The process is explained in Section bla.

\section{Scope \& Limitations}
The feature set of the implemented app is representative for most, but not every type of app (see Section \ref{section::feature_selection}). 
Therefore the results cannot be generalized to every type of possible app. 
However, the results should be seen as indicators of Flutters value as a cross-platform framework for the 
archetypal mobile app. 
Furthermore, the deductively chosen methodology yields the potential of finding adjacent hypothesis which may be
further explored by other researchers.\\
Additionally, the usability study doesn't have a statistical significance due to its qualitative nature. Nevertheless, the depth of detail
in expert interviews is much greater compared quantitative methods, and unthought of considerations may be suggested by the interviewees.\\
This research attribute is especially compelling given the current research state on the Flutter framework as mentioned above. 

\section{Following Chapter Summary}
\textit{- TODO: Describe structure of thesis and summarize each chapter. 
This should be somehow interwoven into introduction.}