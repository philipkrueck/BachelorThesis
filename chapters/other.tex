
%%%%% CHAPTER: INTRODUCTION %%%%%%%%%%%%%%%%%%%%%%%%%%%%%%%%%%%%%%%%%%%%%%%%%%%%
\chapter{Introduction}

%%%%% SECTION: BACKGROUND %%%%%%%%%%%%%%%%%%%%%%%%%%%%%%%%%%%%%%%%%%%%%%%%%%%%
\section{Background}
TODO: give a background on native, cross-platform and Flutters development approach


%%%%% SECTION: BUSINESS CONTEXT %%%%%%%%%%%%%%%%%%%%%%%%%%%%%%%%%%%%%%%%%%%%%%%%%%%%
\section{Business Context}
TODO: give background on apploft and why this thesis is relevant in a business context

%%%%% SECTION: OBJECTIVES AND RESEARCH QUESTIONS %%%%%%%%%%%%%%%%%%%%%%%%%%%%%%%%%%%%%%%%%%%%
\section{Objectives and Research Questions}
RQ 1: Is Flutter a viable alternative for developing a typical medium-sized mobile application?\\
RQ 2: How performant is Flutter in terms of CPU, GPU and memory usage as well as application size compared to a native iOS application?\\
RQ 3: What is the difference of Flutter and iOS development in terms of development complexity?\\



%%%%% SECTION: METHODS %%%%%%%%%%%%%%%%%%%%%%%%%%%%%%%%%%%%%%%%%%%%%%%%%%%%%%%%%%%%%%%%%%
\section{Methods}
\subsection{Theoretical background & Literature Review}
- Theoretical background information and decision considerations are organized through a literature review\\
- Information on subject is collected using online search engines and data bases such as Google Scholar, Sci-hub and IEEE Xplore\\
- Flutter is so new (first stable release in 2018) that it is difficult to find peer-reviewed articles on the subject\\
- Keywords for searching include ...\\
- Information about Fluter specifically was gathered using Gooles official documenation on Flutter and its associated programming language Dart\\
- For further elaboration on specific topics, information is searched through the web and the quality and relevance of is cross examined before use. These include professional blogs and developer community websites.\\

\subsection{RQ1}
- specific prototype development to test proof of concept
\subsection{RQ2}
- The performance analysis is conducted by comparing X tests for each performance metric, on both the Flutter and iOS app\\
- Use mean as measurement of tests for comparison purposes\\
- TODO: Explain which tools etc. were used for comparison (Xcode Tools + Flutter Profiling)\\
\subsection{RQ3}
- Main measurement of code complexity comparison would be Lines of Code in both the iOS and the Flutter app.\\
\textit{- This is quite basic - I will not be able to elaborate too much on this. Maybe this should be part of RQ2?}

%%%%% SECTION: SCOPE + LIMITATION %%%%%%%%%%%%%%%%%%%%%%%%%%%%%%%%%%%%%%%%%%%%%%%%%%%%%%%%
\section{Scope and Limitation}
- RQ1 can actually only be answered for specific app and conclusion cannot be generalized\\
- RQ2 Performance comparison is bound to specific app which includes relevant and typical, but not all general mobile use cases\\
- Therefore, the performance analysis should not be considered to be generalizable to all types of mobile applications\\
- This thesis does not aim to make generalizable claims, but rather evaluate whether Flutter could potentially be a valid alternative for mobile development and if it should be studied further\\

%%%%% SECTION: THESIS STRUCTURE %%%%%%%%%%%%%%%%%%%%%%%%%%%%%%%%%%%%%%%%%%%%%%%%%%%%%%%%
\section{Thesis Structure}
- TODO: Describe structure of thesis and summarize each chapter.


%%%%% CHAPTER: LITERATURE REVIEW %%%%%%%%%%%%%%%%%%%%%%%%%%%%%%%%%%%%%%%%%%%%%%%%%%%%
\chapter{Literature Review}

Was sind die cutting edge findings von anderen researchern? \\
Meinen Beitrag dahingehend erläutern \\
Unbeantwortete Fragen darlegen und Fokus auf die Research Gap meiner Arbeit legen

%%%%% CHAPTER: THEORETICAL BACKGROUND %%%%%%%%%%%%%%%%%%%%%%%%%%%%%%%%%%%%%%%%%%%%%%%%%%%%
\chapter{Theoretical Background}
Why Flutter? How does Flutter work?
- Dartlang -> programming paradigms\\
- Dart VM\\
- JIT and AOT\\
- Cross Compilation\\ 
- Hot Reload\\
- Native performance due to architectural-level performance optimization\\
- Widget tree, Element tree and Render Object tree\\ 
- Rendering Engine + Widget Tree Rendering (Skia)\\
- Widgets\\
- Stateful and Stateless Widgets\\
- Material and Cupertino Design Widgets\\
- State Management + declarative UI\\ 
- setState() + Ephemeral States\\
- Shared App State -> ScopedMode, BLoC, Redux, Provider\\
Background on traditional native iOS app development\\


%%%%% KAPITEL: APPLICATION DESIGN %%%%%%%%%%%%%%%%%%%%%%%%%%%%%%%%%%%%%%%%%%%%%%%%%%%%
\chapter{Application Design}
This thesis does not cover the entire development of the application. It only persents the necessary parts for the purposes of later comparison between the native application.\\
- Requirements (Functional, Non-functional requirements)\\
- UI\\
- Flutter App Architecture\\
- Widget Architecture of Screens\\ 
- Navigation\\
- Utilized package dependencies\\
- API Access and Networking\\
- State Management\\




%%%%% KAPITEL: COMPARATIVE ANALYSIS %%%%%%%%%%%%%%%%%%%%%%%%%%%%%%%%%%%%%%%%%%%%%%%%%%%%
\chapter{Comparative Analysis}
\subsection{Performance Evaluation}
- look at specific actions within app (like scrolling, opening page) for comparison\\
- use "Instruments" from Xcode\\
\subsubsection{CPU usage}
-> Time profiler tool for CPU usage\\
\subsubsection{GPU usage}
-> Core Animation tool for GPU FPS\\
\subsubsection{memory usage}
-> Allocations tool for memory usage\\
\subsubsection{energy usage? basically results from above metrics}
\subsection{Code Complexity}
- LOC\\
- What else?\\





%%%%% KAPITEL: DISCUSSION %%%%%%%%%%%%%%%%%%%%%%%%%%%%%%%%%%%%%%%%%%%%%%%%%%%%
\chapter{Discussion}
Key Findings presentieren und beurteilen in Bezug auf vorige Berücksichtigungen\\
Bedeutung der Findings interpretieren



%%%%% KAPITEL: ZUSAMMENFASSUNG %%%%%%%%%%%%%%%%%%%%%%%%%%%%%%%%%%%%%%%%%%%%%%%%%%%%
\chapter{Summary}
Zusammenfassung der Key Findings sowie deren Signifikanz und Implikationen\\
Kritische Beurteilung möglicher Einschränkungen meiner Forschung\\
Zukünftige Forschungsfragen\\
Outlook\\


